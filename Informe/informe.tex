\documentclass{article}  % Define la clase del documento.

% Paquetes de idioma y codificación
\usepackage[utf8]{inputenc}
\usepackage[T1]{fontenc}
\usepackage[spanish]{babel}  % Ajusta el idioma del documento a español.
\addto\captionsspanish{
  \renewcommand{\figurename}{Figura}
  \renewcommand{\tablename}{Tabla}
}

\usepackage{tabularx}  % Permite la creación de tablas con ancho ajustable.

\usepackage{caption}
\usepackage{subcaption}

% Paquete de geometría para configurar márgenes y tamaño de papel
\usepackage[letterpaper, margin=3cm]{geometry}

% Paquetes de tipografía
\usepackage{mathptmx}    % Usa Times New Roman como fuente.
\usepackage{microtype}   % Mejora la justificación del texto.

% Paquetes para manejo de colores y gráficos
\usepackage{xcolor}      % Define y utiliza colores.
\usepackage{graphicx}    % Permite la inserción de imágenes.
\usepackage{tikz}        % Creación de gráficos vectoriales.

% Configuración de enlaces y referencias cruzadas


\usepackage{media9} % Permite la inserción de multimedia.

% Paquetes para la mejora visual de tablas y figuras
\usepackage{booktabs}    % Para tablas de alta calidad.
\usepackage{float}       % Controla la posición de figuras y tablas.

% Paquete para la personalización de códigos fuente
\usepackage{listings}
\lstset{
    literate=
    {á}{{\'a}}1 {é}{{\'e}}1 {í}{{\'i}}1 {ó}{{\'o}}1 {ú}{{\'u}}1
    {Á}{{\'A}}1 {É}{{\'E}}1 {Í}{{\'I}}1 {Ó}{{\'O}}1 {Ú}{{\'U}}1
    {ñ}{{\~n}}1 {Ñ}{{\~N}}1 {ü}{{\"u}}1 {Ü}{{\"U}}1,
    backgroundcolor=\color{backcolour},
    commentstyle=\color{codegreen},
    keywordstyle=\color{codepurple},
    numberstyle=\tiny\color{codegray},
    stringstyle=\color{red},
    basicstyle=\ttfamily\small,
    breakatwhitespace=false,
    breaklines=true,
    captionpos=b,
    keepspaces=true,
    numbers=left,
    numbersep=5pt,
    showspaces=false,
    showstringspaces=false,
    showtabs=false,
    tabsize=2,
    language=TeX,
    morecomment=[l]\#,
    frame=single,
    rulecolor=\color{black}
}

% Definición de colores al estilo Visual Studio Code
\definecolor{darkblue}{rgb}{0.0, 0.0, 0.55}  % Enlaces
\definecolor{codegreen}{rgb}{0.25, 0.49, 0.48}  % Comentarios
\definecolor{codegray}{rgb}{0.5, 0.5, 0.5}  % Números y anotaciones
\definecolor{codepurple}{rgb}{0.58, 0, 0.82}  % Palabras clave
\definecolor{backcolour}{rgb}{0.95, 0.95, 0.92}  % Fondo de código

% Configuraciones de párrafo y matemáticas
\usepackage{amsmath}
\usepackage{parskip}    % Espaciado entre párrafos.
\usepackage{ragged2e}   % Justificación mejorada.
\usepackage{multicol}
\usepackage{multirow}   % Para multirow
\usepackage{float}      % Para la opción [H] en las tablas/figuras


% Configuración de secciones y encabezados
\usepackage{titlesec}
\titleclass{\part}{top} % Make part like a class
\titleformat{\part}[display]
  {\normalfont\huge\bfseries\centering}{\thepart}{40pt}{\Huge}
\titlespacing*{\part}{0pt}{-60pt}{10pt}
\titleformat{\part}
  {\normalfont\huge\bfseries}{}{0pt}{}

% Asegúrate de usar esto para mantener el estilo en las páginas de las partes
\titleformat{\part}[display]
  {\normalfont\huge\bfseries}{}{0pt}{}
  [\thispagestyle{fancy}] % Aplica el estilo fancy a las páginas de las partes

% Configuración de encabezados y pies de página personalizados
\usepackage{fancyhdr}
\pagestyle{fancy}
\fancyhf{}
\fancyhead[L]{\raisebox{0.20cm}{\textbf{Tecnología del Hormigón}}}
\fancyhead[R]{\raisebox{0.1cm}{\includegraphics[width=0.25\linewidth]{LOGO_UNIVERSIDAD.jpg}}}
\fancyhead[C]{\rule{\textwidth}{0.6pt}}
\fancyfoot[C]{\rule{\textwidth}{0.6pt}}
\fancyfoot[R]{\raisebox{-1.5\baselineskip}{\thepage}}
\renewcommand{\headrulewidth}{0pt}
\renewcommand{\footrulewidth}{0pt}

% Configuración avanzada de geometría
\geometry{
  top=3.5cm, % Aumenta el espacio en la parte superior para subir el encabezado
  bottom=2.5cm,
  headheight=2.5cm % Aumenta la altura del encabezado si es necesario
}

% Configuracion de bibliografia
\usepackage{natbib}
\usepackage{hyperref}
\hypersetup{
    colorlinks   = true,
    linkcolor    = darkblue,
    citecolor    = black,
    filecolor    = blue,
    urlcolor     = blue
}

\begin{document}
%----------------------------------------------------------------------------------------
% PORTADA
%----------------------------------------------------------------------------------------
\begin{titlepage}%Inicio de la carátula, solo modificar los datos necesarios
\newcommand{\HRule}{\rule{\linewidth}{0.5mm}} 
\center 
%----------------------------------------------------------------------------------------
%	ENCABEZADO
%----------------------------------------------------------------------------------------
\includegraphics[width=10cm]{LOGO_UNIVERSIDAD.jpg}\\ % Si esta plantilla se copio correctamente, va a llevar la imagen del logo de la facultad.OBS: Es necesario incluir el paquete: graphicx
\vspace{3cm}
%----------------------------------------------------------------------------------------
%	SECCION DEL TITULO
%----------------------------------------------------------------------------------------
\HRule \\[0.4cm]
{ \huge \bfseries Tecnología del Hormigón}\\[0.4cm] % Titulo del documento
{ \huge \bfseries Taller 4 - Hormigón Autocompactante - Parte 2}\\[0.4cm] % Titulo del documento
\HRule \\[1.5cm]
 \vspace{5cm}
%----------------------------------------------------------------------------------------
%	SECCION DEL AUTOR
%----------------------------------------------------------------------------------------
\begin{flushright}
  { \textbf{Profesor:}\\
  Alvaro Paul\\
  \textbf{Ayudante:}\\
  Felipe Ronda\\
  \textbf{Alumnos:} \\
  Felipe Vicencio\\
  Lukas Wolff\\
}
\end{flushright}
\vspace{1cm}
%----------------------------------------------------------------------------------------
%	SECCION DE LA FECHA
%----------------------------------------------------------------------------------------
{\large \textbf{\today}}\\[2cm] % El comando \today coloca la fecha del dia, y esto se actualiza con cada compilacion, en caso de querer tener una fecha estatica, reemplazar el \today por la fecha deseada
\end{titlepage}

\newpage
\thispagestyle{empty} % Deshabilita el número de página en la página del índice

%----------------------------------------------------------------------------------------
%  INDICE
%----------------------------------------------------------------------------------------
%\newpage
%\thispagestyle{empty} % Deshabilita el número de página en la página del índice
%\tableofcontents
%\thispagestyle{plain} % Deshabilita el encabezado en la página del índice
%\thispagestyle{empty} % Deshabilita el número de página en la página del índice
%\newpage

%\newpage
%\thispagestyle{empty}
%\listoffigures 
%\thispagestyle{plain} % Deshabilita el encabezado en la página del índice %
%\thispagestyle{empty}
%----------------------------------------------------------------------------------------
%ACÁ EMPIEZA EL INFORME
\setcounter{page}{1}
%----------------------------------------------------------------------------------------

\section{Desarrollo}

\subsubsection*{Pregunta 1}

Sea la tabla:

\begin{table}[H]
\centering
\caption{Dosificaciones de mezcla para Hormigones 1 y 2}
\begin{tabular}{|l|c|c|}
\hline
\textbf{Material} & \textbf{Hormigón 1} & \textbf{Hormigón 2} \\ \hline
Cemento (kg/m$^3$) & 450 & 350 \\ \hline
Agua (kg/m$^3$) & 180 & 180 \\ \hline
Árido Grueso (kg/m$^3$) & 750 & 1050 \\ \hline
Árido Fino (kg/m$^3$) & 850 & 750 \\ \hline
Aditivo Reductor de Agua (SP) & 1.25\% & 0.45\% \\ \hline
\end{tabular}
\end{table}

Basadondose en la mayor cantidad de arido fino, menor cantidad de arido grueso, asi como un mayor contenido de aditivo reductor de agua, se puede concluir que el Hormigón 1 es el que corresponde a un Hormigón Autocompactante (HAC), mientras que el Hormigón 2 corresponde a un Hormigón Convencional.

Estas diferencias son debido a los requisitos de trabajabilidad y fluidez que debe cumplir un HAC, los cuales se logran mediante:

\begin{itemize}
  \item Mayor cantidad de árido fino: Esto ayuda a mejorar la cohesión y la fluidez del hormigón.
  \item Menor cantidad de árido grueso: Esto facilita el flujo del hormigón y reduce la segregación.
  \item Mayor contenido de aditivo reductor de agua: Esto mejora la trabajabilidad del hormigon sin aumentar la cantidad de agua, lo que es crucial para mantener la resistencia y durabilidad del HAC.
\end{itemize}

Ahora bien, estos ajustes tienen implicaciones en las propiedades del hormigón en estado endureciado tambien, las cuales son:

\begin{itemize}
  \item Mayor cantidad de arido fino: Puede aumentar la densidad y reducir la porosidad del hormigón, mejorando su resistencia y durabilidad. 
  \item Menor cantidad de árido grueso: Se mejora la homogeneidad del hormigón, asi como la interfaz pasta-cemento, lo que resulta en una mejor resistencia.
  \item \item Mayor contenido de aditivo reductor de agua: Esto puede mejorar la resistencia del hormigón al reducir la relación agua-cemento efectiva, lo que resulta en una matriz más densa y fuerte.
\end{itemize}

De esta forma, se puede concluir que el HAC es un hormigon que en estado fresco presenta una alta fluidez y trabajabilidad, mientras que en estado endurecido logra una mayor densidad, resistencia y durabilidad en comparación con un hormigón convencional.

\subsection*{Pregunta 2}

Un HAC correctamente elaborado deberia mostrar los siguientes signos:

\begin{itemize}
  \item Fluidez y Autocompactación: El HAC debe ser capaz de fluir y compactarse por sí mismo bajo su propio peso, llenando los moldes sin necesidad de vibración externa.
  \item Resistencia a la Segregación: El HAC debe mantener una mezcla homogénea sin que los componentes se separen durante el vertido o en el ensayo de cono invertido o J-Ring.
  \item Resistencia a la Exudación: El HAC debe evitar la liberación excesiva de agua en la superficie, lo que puede afectar negativamente la calidad del acabado, lo cual se puede evaluar mediante el ensayo de J-Ring.
\end{itemize}

Estos ensayos son relativamente simples de realizar en obra y proporcionan una buena indicación de si el HAC cumple con los requisitos necesarios para su aplicación.

Segun lo visto en el taller, en el ensayo de cono invertido, un HAC correctamente elaborado deberia mostrar un diametro final mayor a 650 mm, indicando una buena fluidez y capacidad de autocompactación y menor a 800 mm para evitar problemas de segregación.

La siguiente figura muetra la diferencia entre un hormigon extremadamente fluido y uno con una fluidez casi optima (en el taller no se llego a ver un hormigon con fluidez optima en el ensayo de cono invertido, pero se puede observar la diferencia entre uno muy fluido y otro menos fluido).:


\begin{figure}[H]
\centering
\begin{subfigure}[b]{0.45\textwidth}
    \centering
    \includegraphics[width=\textwidth]{Imagenes/a.png}
    \caption{HAC con fluidez casi optima.}
    \label{fig:imagen1}
\end{subfigure}
\hfill
\begin{subfigure}[b]{0.45\textwidth}
    \centering
    \includegraphics[width=\textwidth]{Imagenes/b.png}
    \caption{HAC con fluidez extremadamente alta.}
    \label{fig:imagen2}
\end{subfigure}
\label{fig:comparacion}
\end{figure}

\subsection*{Pregunta 4}


El HAC requiere un diseño reológico con baja tensión de corte y una viscosidad plástica que evite la segregación de la mezcla. Para esto, se emplea la ayuda de aditivos químicos que modifican parámetros como la reología, cohesión, y desarrollo de hidratación. Los aditivos más comunes se muestran a continuación.

\subsubsection*{1. Superplastificantes (SP): Reductores de agua}

Los superplastificantes, en especial los policarboxilatos, son el aditivo principal del HAC. Funcionan reduciendo la tensión de corte a fluir del material, permitiendo que la mezcla fluya por gravedad sin necesidad de vibración. Según ACI 237R, los policarboxilatos actúan por dos mecanismos:

\begin{itemize}
    \item Dispersión electrostática: los grupos aniónicos del polímero generan repulsión entre partículas de cemento.
    \item Efecto estérico: las cadenas laterales generan separación física entre partículas, estabilizando la suspensión.
\end{itemize}

Reológicamente, se disminuye la viscosidad aparente y aumentan la fluidez, aumentando la \textit{filling ability} y la \textit{passing ability}. 

Consecuencias de una dosificación inadecuada:
\begin{itemize}
    \item Por defecto: disminución de fluidez, pérdida de autocompactación y mayor riesgo de bloqueo en zonas confinadas.
    \item Por exceso: segregación, exudación, pérdida de cohesión y riesgo de asentamientos diferenciales en elementos altos.
\end{itemize}

\subsubsection*{2. Modificadores de viscosidad (VMA)}

Los VMA controlan la viscosidad plástica y mejoran la estabilidad del HAC, principalmente en mezclas con baja cantidad de finos. El ACI 237R muestra que estos aditivos forman una red coloidal que aumenta la cohesión interna sin afectar significativamente la fluidez inicial. 

Su uso incrementa la viscosidad de la mezcla y reduce la segregación dinámica, sedimentación de áridos gruesos y exudación.

Consecuencias de una dosificación inadecuada:
\begin{itemize}
    \item Por defecto: pérdida de estabilidad, aparición de segregación y falta de homogeneidad.
    \item Por exceso: mezcla excesivamente viscosa, reducción de fluidez y disminución de la capacidad de pase entre armaduras.
\end{itemize}

\subsubsection*{3. Retardadores de fraguado}

Dado el alto contenido de finos y aditivos del HAC, el fraguado puede acelerarse. Para que no ocurra se usan retardadores, que permiten controlar el inicio de fraguado y aumentar el tiempo de trabajabilidad, lo cual es importante durante el transporte o en climas áridos.

Químicamente, actúan por adsorción sobre partículas de cemento, retrasando la hidratación de las fases C$_3$A y C$_3$S.

Consecuencias de una dosificación inadecuada:
\begin{itemize}
    \item Exceso: retraso excesivo del fraguado, riesgo de deformaciones plásticas y exudación tardía.
    \item Defecto: pérdida de asentamiento y endurecimiento prematuro.
\end{itemize}

\subsubsection*{4. Acelerantes}

En climas fríos o en elementos prefabricados se pueden añadir acelerantes para aumentar la velocidad inicial de hidratación y obtener resistencias tempranas más altas.

Consecuencias de una dosificación inadecuada:
\begin{itemize}
    \item Exceso: incremento del calor de hidratación, riesgo de fisuración térmica.
    \item Defecto: fraguado lento y baja resistencia en primeras horas, especialmente crítico en desmolde.
\end{itemize}




%\bibliographystyle{unsrtnat}  % o plainnat
%\bibliography{ref}  % Nombre del archivo .bib

\end{document}