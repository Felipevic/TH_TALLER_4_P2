\documentclass{article}  % Define la clase del documento.

% Paquetes de idioma y codificación
\usepackage[utf8]{inputenc}
\usepackage[T1]{fontenc}
\usepackage[spanish]{babel}  % Ajusta el idioma del documento a español.
\addto\captionsspanish{
  \renewcommand{\figurename}{Figura}
  \renewcommand{\tablename}{Tabla}
}

\usepackage{tabularx}  % Permite la creación de tablas con ancho ajustable.

\usepackage{caption}
\usepackage{subcaption}

% Paquete de geometría para configurar márgenes y tamaño de papel
\usepackage[letterpaper, margin=3cm]{geometry}

% Paquetes de tipografía
\usepackage{mathptmx}    % Usa Times New Roman como fuente.
\usepackage{microtype}   % Mejora la justificación del texto.

% Paquetes para manejo de colores y gráficos
\usepackage{xcolor}      % Define y utiliza colores.
\usepackage{graphicx}    % Permite la inserción de imágenes.
\usepackage{tikz}        % Creación de gráficos vectoriales.

% Configuración de enlaces y referencias cruzadas


\usepackage{media9} % Permite la inserción de multimedia.

% Paquetes para la mejora visual de tablas y figuras
\usepackage{booktabs}    % Para tablas de alta calidad.
\usepackage{float}       % Controla la posición de figuras y tablas.

% Paquete para la personalización de códigos fuente
\usepackage{listings}
\lstset{
    literate=
    {á}{{\'a}}1 {é}{{\'e}}1 {í}{{\'i}}1 {ó}{{\'o}}1 {ú}{{\'u}}1
    {Á}{{\'A}}1 {É}{{\'E}}1 {Í}{{\'I}}1 {Ó}{{\'O}}1 {Ú}{{\'U}}1
    {ñ}{{\~n}}1 {Ñ}{{\~N}}1 {ü}{{\"u}}1 {Ü}{{\"U}}1,
    backgroundcolor=\color{backcolour},
    commentstyle=\color{codegreen},
    keywordstyle=\color{codepurple},
    numberstyle=\tiny\color{codegray},
    stringstyle=\color{red},
    basicstyle=\ttfamily\small,
    breakatwhitespace=false,
    breaklines=true,
    captionpos=b,
    keepspaces=true,
    numbers=left,
    numbersep=5pt,
    showspaces=false,
    showstringspaces=false,
    showtabs=false,
    tabsize=2,
    language=TeX,
    morecomment=[l]\#,
    frame=single,
    rulecolor=\color{black}
}

% Definición de colores al estilo Visual Studio Code
\definecolor{darkblue}{rgb}{0.0, 0.0, 0.55}  % Enlaces
\definecolor{codegreen}{rgb}{0.25, 0.49, 0.48}  % Comentarios
\definecolor{codegray}{rgb}{0.5, 0.5, 0.5}  % Números y anotaciones
\definecolor{codepurple}{rgb}{0.58, 0, 0.82}  % Palabras clave
\definecolor{backcolour}{rgb}{0.95, 0.95, 0.92}  % Fondo de código

% Configuraciones de párrafo y matemáticas
\usepackage{amsmath}
\usepackage{parskip}    % Espaciado entre párrafos.
\usepackage{ragged2e}   % Justificación mejorada.
\usepackage{multicol}
\usepackage{multirow}   % Para multirow
\usepackage{float}      % Para la opción [H] en las tablas/figuras


% Configuración de secciones y encabezados
\usepackage{titlesec}
\titleclass{\part}{top} % Make part like a class
\titleformat{\part}[display]
  {\normalfont\huge\bfseries\centering}{\thepart}{40pt}{\Huge}
\titlespacing*{\part}{0pt}{-60pt}{10pt}
\titleformat{\part}
  {\normalfont\huge\bfseries}{}{0pt}{}

% Asegúrate de usar esto para mantener el estilo en las páginas de las partes
\titleformat{\part}[display]
  {\normalfont\huge\bfseries}{}{0pt}{}
  [\thispagestyle{fancy}] % Aplica el estilo fancy a las páginas de las partes

% Configuración de encabezados y pies de página personalizados
\usepackage{fancyhdr}
\pagestyle{fancy}
\fancyhf{}
\fancyhead[L]{\raisebox{0.20cm}{\textbf{Tecnología del Hormigón}}}
\fancyhead[R]{\raisebox{0.1cm}{\includegraphics[width=0.25\linewidth]{LOGO_UNIVERSIDAD.jpg}}}
\fancyhead[C]{\rule{\textwidth}{0.6pt}}
\fancyfoot[C]{\rule{\textwidth}{0.6pt}}
\fancyfoot[R]{\raisebox{-1.5\baselineskip}{\thepage}}
\renewcommand{\headrulewidth}{0pt}
\renewcommand{\footrulewidth}{0pt}

% Configuración avanzada de geometría
\geometry{
  top=3.5cm, % Aumenta el espacio en la parte superior para subir el encabezado
  bottom=2.5cm,
  headheight=2.5cm % Aumenta la altura del encabezado si es necesario
}

% Configuracion de bibliografia
\usepackage{natbib}
\usepackage{hyperref}
\hypersetup{
    colorlinks   = true,
    linkcolor    = darkblue,
    citecolor    = black,
    filecolor    = blue,
    urlcolor     = blue
}

\begin{document}
%----------------------------------------------------------------------------------------
% PORTADA
%----------------------------------------------------------------------------------------
\begin{titlepage}%Inicio de la carátula, solo modificar los datos necesarios
\newcommand{\HRule}{\rule{\linewidth}{0.5mm}} 
\center 
%----------------------------------------------------------------------------------------
%	ENCABEZADO
%----------------------------------------------------------------------------------------
\includegraphics[width=10cm]{LOGO_UNIVERSIDAD.jpg}\\ % Si esta plantilla se copio correctamente, va a llevar la imagen del logo de la facultad.OBS: Es necesario incluir el paquete: graphicx
\vspace{3cm}
%----------------------------------------------------------------------------------------
%	SECCION DEL TITULO
%----------------------------------------------------------------------------------------
\HRule \\[0.4cm]
{ \huge \bfseries Tecnología del Hormigón}\\[0.4cm] % Titulo del documento
{ \huge \bfseries Taller 4 - Hormigón Autocompactante - Parte 2}\\[0.4cm] % Titulo del documento
\HRule \\[1.5cm]
 \vspace{5cm}
%----------------------------------------------------------------------------------------
%	SECCION DEL AUTOR
%----------------------------------------------------------------------------------------
\begin{flushright}
  { \textbf{Profesor:}\\
  Alvaro Paul\\
  \textbf{Ayudante:}\\
  Felipe Ronda\\
  \textbf{Alumnos:} \\
  Felipe Vicencio\\
  Lukas Wolff\\
}
\end{flushright}
\vspace{1cm}
%----------------------------------------------------------------------------------------
%	SECCION DE LA FECHA
%----------------------------------------------------------------------------------------
{\large \textbf{\today}}\\[2cm] % El comando \today coloca la fecha del dia, y esto se actualiza con cada compilacion, en caso de querer tener una fecha estatica, reemplazar el \today por la fecha deseada
\end{titlepage}

\newpage
\thispagestyle{empty} % Deshabilita el número de página en la página del índice

%----------------------------------------------------------------------------------------
%  INDICE
%----------------------------------------------------------------------------------------
%\newpage
%\thispagestyle{empty} % Deshabilita el número de página en la página del índice
%\tableofcontents
%\thispagestyle{plain} % Deshabilita el encabezado en la página del índice
%\thispagestyle{empty} % Deshabilita el número de página en la página del índice
%\newpage

%\newpage
%\thispagestyle{empty}
%\listoffigures 
%\thispagestyle{plain} % Deshabilita el encabezado en la página del índice %
%\thispagestyle{empty}
%----------------------------------------------------------------------------------------
%ACÁ EMPIEZA EL INFORME
\setcounter{page}{1}
%----------------------------------------------------------------------------------------

\section{Desarrollo}

\subsection*{Pregunta 1}

Dada la siguiente tabla:

\begin{table}[H]
\centering
\caption{Dosificaciones de mezcla para Hormigones 1 y 2}
\begin{tabular}{|l|c|c|}
\hline
\textbf{Material} & \textbf{Hormigón 1} & \textbf{Hormigón 2} \\ \hline
Cemento (kg/m$^3$) & 450 & 350 \\ \hline
Agua (kg/m$^3$) & 180 & 180 \\ \hline
Árido Grueso (kg/m$^3$) & 750 & 1050 \\ \hline
Árido Fino (kg/m$^3$) & 850 & 750 \\ \hline
Aditivo Reductor de Agua (SP) & 1.25\% & 0.45\% \\ \hline
\end{tabular}
\end{table}

Basándose en la mayor cantidad de árido fino, menor proporción de árido grueso y mayor contenido de aditivo reductor de agua, se puede deducir que el Hormigón 1 corresponde a un Hormigón Autocompactante (HAC), mientras que el Hormigón 2 corresponde a un Hormigón Convencional.

Estas diferencias se explican en función de los requisitos de trabajabilidad y fluidez propios de un HAC, los cuales se logran mediante:

\begin{itemize}
\item \textbf{Mayor cantidad de árido fino}: Aumenta la cohesión y fluidez del hormigón, evitando la segregación.
\item \textbf{Menor cantidad de árido grueso}: Facilita el flujo del hormigón entre los elementos del encofrado y las armaduras.
\item \textbf{Mayor contenido de aditivo reductor de agua}: Mejora la trabajabilidad sin incrementar la relación agua/cemento, manteniendo así la resistencia y durabilidad.
\end{itemize}

En cuanto a las propiedades del hormigón endurecido, estos ajustes generan efectos en:

\begin{itemize}
\item \textbf{Mayor proporción de árido fino}: Disminuye la porosidad y aumenta la densidad, lo que mejora la resistencia y durabilidad.
\item \textbf{Menor proporción de árido grueso}: Favorece una matriz más homogénea y una mejor adherencia pasta-agregado, contribuyendo a una mayor resistencia mecánica.
\item \textbf{Mayor contenido de aditivo reductor de agua}: Permite una menor relación agua/cemento efectiva, lo que se traduce en una matriz más densa y resistente.
\end{itemize}

En conclusión, el Hormigón Autocompactante (HAC) presenta en su estado fresco una alta fluidez y capacidad de llenado sin necesidad de vibración. En estado endurecido, exhibe una mayor densidad, resistencia mecánica y durabilidad en comparación con un hormigón convencional.

\subsection*{Pregunta 2}

Un Hormigón Autocompactante (HAC) correctamente elaborado debe presentar las siguientes características en estado fresco:

\begin{itemize}
\item \textbf{Fluidez y autocompactación}: El hormigón debe ser capaz de fluir y compactarse por sí mismo bajo su propio peso, llenando completamente los moldes sin necesidad de vibración externa.
\item \textbf{Resistencia a la segregación}: El HAC debe mantener una mezcla homogénea sin separar sus componentes durante el vertido o durante la ejecución de ensayos como el cono invertido o el J-Ring.
\item \textbf{Resistencia a la exudación}: No debe liberar agua de forma excesiva hacia la superficie, lo que afectaría el acabado. Este aspecto también puede evaluarse mediante el ensayo de J-Ring.
\end{itemize}

Estos ensayos son relativamente simples de ejecutar en obra y permiten verificar si el HAC cumple con los requisitos necesarios para su correcta aplicación.

Según lo observado en el taller, en el ensayo de cono invertido, un HAC correctamente dosificado debiera presentar un diámetro de extensión mayor a 650 mm, lo que indica una adecuada fluidez y capacidad de autocompactación, pero inferior a 800 mm, para evitar problemas de segregación.

A continuación se muestra una comparación entre un hormigón con fluidez cercana al óptimo y otro con fluidez excesiva (en el taller no se observó un HAC con fluidez óptima en este ensayo, pero es posible apreciar la diferencia entre ambos comportamientos):

\begin{figure}[H]
\centering
\begin{subfigure}[b]{0.45\textwidth}
\centering
\includegraphics[width=\textwidth]{Imagenes/a.png}
\caption{HAC con fluidez cercana al óptimo.}
\label{fig:imagen1}
\end{subfigure}
\hfill
\begin{subfigure}[b]{0.45\textwidth}
\centering
\includegraphics[width=\textwidth]{Imagenes/b.png}
\caption{HAC con fluidez excesiva.}
\label{fig:imagen2}
\end{subfigure}
\caption{Comparación visual de HAC con distintos niveles de fluidez.}
\label{fig:comparacion}
\end{figure}

Según lo comentado durante el taller, el Hormigón Autocompactante (HAC) tiene un costo aproximado de un 60,% mayor que el de un hormigón convencional. Sin embargo, este sobrecosto inicial se justifica en función de las ventajas que ofrece a lo largo del proceso constructivo y durante la vida útil de la estructura, principalmente en los siguientes aspectos:

\begin{itemize}
\item \textbf{Menor mano de obra}: Al no requerir vibración ni compactación manual, se reduce la necesidad de mano de obra especializada, lo que se traduce en ahorros potenciales en costos laborales.
\item \textbf{Mayor rendimiento en colocación}: El HAC permite una colocación más rápida y eficiente, reduciendo los tiempos de construcción y, por ende, los costos indirectos asociados a la obra.
\item \textbf{Mejor calidad de acabado}: Su capacidad para fluir y compactarse de forma homogénea mejora la calidad de los acabados superficiales, disminuyendo la necesidad de reparaciones o retrabajos.
\item \textbf{Mayor durabilidad y vida útil}: Debido a su menor porosidad y mayor densidad, el HAC exhibe mejor resistencia frente a agentes agresivos, lo que aumenta su durabilidad y reduce los costos de mantenimiento a largo plazo.
\item \textbf{Reducción de defectos estructurales}: La homogeneidad y cohesión que ofrece el HAC disminuyen la probabilidad de defectos relacionados con oquedades o falta de consolidación, reduciendo así los costos asociados a reparaciones o refuerzos posteriores.
\end{itemize}

En consecuencia, aunque el costo inicial del HAC es mayor, los beneficios económicos y técnicos que se obtienen incluso durante la etapa constructiva justifican su uso en proyectos donde la calidad, durabilidad y eficiencia constructiva sean prioritarias.

\subsection*{Pregunta 4}

El HAC requiere un diseño reológico con baja tensión de cedencia y una viscosidad plástica suficiente para evitar la segregación de la mezcla. Para esto, se emplea la ayuda de aditivos químicos que modifican parámetros como la reología, cohesión y desarrollo de hidratación. Los aditivos más comunes se muestran a continuación \citep{ACI237R07}.

\subsubsection*{1. Superplastificantes (SP): Reductores de agua}

Los superplastificantes, en especial los policarboxilatos, son el aditivo principal del HAC. Funcionan reduciendo la tensión de cedencia necesaria para que el material comience a fluir, permitiendo que la mezcla fluya por gravedad sin necesidad de vibración. Según el ACI 237R, los policarboxilatos actúan por dos mecanismos:

\begin{itemize}
    \item Dispersión electrostática: los grupos aniónicos del polímero generan repulsión entre partículas de cemento.
    \item Efecto estérico: las cadenas laterales generan separación física entre partículas, estabilizando la suspensión.
\end{itemize}

Reológicamente, se disminuye la viscosidad aparente y se aumenta la fluidez, elevando la filling ability y la passing ability del hormigón autocompactante.

Consecuencias de una dosificación inadecuada:
\begin{itemize}
    \item Por defecto: disminución de fluidez, pérdida de autocompactación y mayor riesgo de bloqueo en zonas confinadas.
    \item Por exceso: segregación, exudación, pérdida de cohesión y riesgo de asentamientos diferenciales en elementos altos.
\end{itemize}

\subsubsection*{2. Modificadores de viscosidad (VMA)}

Los VMA controlan la viscosidad plástica y mejoran la estabilidad del HAC, principalmente en mezclas con baja cantidad de finos. El ACI 237R indica que estos aditivos forman una red coloidal que aumenta la cohesión interna sin afectar significativamente la fluidez inicial.

Su uso incrementa la viscosidad de la mezcla y reduce la segregación dinámica, la sedimentación de áridos gruesos y la exudación.

Consecuencias de una dosificación inadecuada:
\begin{itemize}
    \item Por defecto: pérdida de estabilidad, aparición de segregación y falta de homogeneidad.
    \item Por exceso: mezcla excesivamente viscosa, reducción de fluidez y disminución de la capacidad de pase entre armaduras.
\end{itemize}

\subsubsection*{3. Retardadores de fraguado}

Dado el alto contenido de finos y aditivos del HAC, el fraguado puede acelerarse. Para evitarlo se usan retardadores, que permiten controlar el inicio de fraguado y aumentar el tiempo de trabajabilidad, lo cual es importante durante el transporte o en climas cálidos.

Químicamente, actúan por adsorción sobre partículas de cemento, retrasando la hidratación de las fases C$_3$A y C$_3$S.

Consecuencias de una dosificación inadecuada:
\begin{itemize}
    \item Exceso: retraso excesivo del fraguado, riesgo de deformaciones plásticas y exudación tardía.
    \item Defecto: pérdida de asentamiento y endurecimiento prematuro.
\end{itemize}

\subsubsection*{4. Acelerantes}

En climas fríos o en elementos prefabricados se pueden añadir acelerantes para aumentar la velocidad inicial de hidratación y obtener resistencias tempranas más altas.

Consecuencias de una dosificación inadecuada:
\begin{itemize}
    \item Exceso: incremento del calor de hidratación y riesgo de fisuración térmica.
    \item Defecto: fraguado lento y baja resistencia en primeras horas, especialmente crítico en desmolde.
\end{itemize}

\subsection*{Pregunta 5}

\subsubsection*{Morteros de reparación}

Los morteros sirven para reparar estructuras de hormigón que presentan deterioro, pérdida de recubrimiento o daño estructural localizado. La idea es reconstruir la forma original del elemento y que retome su desempeño mecánico y su durabilidad \citep{MCGuide2024}.

Algunas ventajas de los morteros es su alta adherencia a superficies, estabilidad dimensional, control de la retracción, buena trabajabilidad y resistencia mecánica temprana y final. Además, están formulados para ser aplicados en posiciones horizontales, verticales y en techos sin presentar escurrimiento ni segregación \citep{MCGuide2024}.

Dentro del proceso constructivo, su función es reconstruir zonas dañadas de hormigón, proteger la armadura expuesta y restituir la capacidad mecánica del elemento. Sus condiciones de aplicación son la preparación adecuada de la mezcla, limpieza y perfilado del hormigón a reparar, saturación superficial sin agua libre y curado posterior para conseguir un buen desempeño \citep{MCGuide2024}.

\subsubsection*{Grout}

El grout tiene la función de rellenar espacios confinados y transmitir cargas de manera uniforme entre dos elementos estructurales. Se utilizan cuando se necesita un material fluido, preciso y sin retracción para asegurar contacto total y continuidad entre superficies \citep{FiveStarGroutFT}.

Los grout son mezclas de alta fluidez, expansión controlada y altas resistencias a la compresión. A diferencia de los morteros de reparación, no están pensados para adherirse a una superficie dañada, sino para fluir entre espacios confinados y transferir las cargas \citep{FiveStarGroutFT, Emcekrete40FT}.

Las condiciones de aplicación de los grout implican superficies limpias, alineación correcta de los elementos, vaciado continuo para evitar cavidades y curado para garantizar su desempeño bajo carga \citep{FiveStarGroutFT}.

\subsubsection*{Ejemplos de uso en obra}

\begin{itemize}
    \item Cuando existe deterioro del hormigón, pérdida de sección o exposición de armaduras, se recomienda utilizar morteros de reparación para restituir la geometría y la capacidad resistente del elemento \citep{MCGuide2024}.
    \item Cuando se requiere nivelar maquinaria, rellenar el espacio bajo placas base, fijar pernos de anclaje o asegurar estructuras prefabricadas, se recomienda utilizar grout debido a su fluidez y capacidad de transferencia de carga \citep{FiveStarGroutFT}.
\end{itemize}





\bibliographystyle{unsrtnat}  % o plainnat
\bibliography{ref}  % Nombre del archivo .bib

\end{document}